\documentclass{article}
\usepackage{amsmath}
\begin{document}
\section{Setting up the problem}
We have three angles, \(x\), \(y\) and \(z\). These can be represented as three
components of a vector:
\[ r = \left\langle x, y, z \right\rangle \]

The magnitude of this vector should be \emph{approximately} equal to the
acceleration due to gravity, \(9.8{\mbox{m/s}}^2\). I can compute the magnitude
of the vector as follows:

\[ |r| = \sqrt{x^2 + y^2 + z^2} \]

In case the magnitude of the vector is different than the acceleration due to
gravity, we should use this value instead. Also, we should report if this value
is outside of some sort of tolerance. Perhaps we'll be a bit liberal on this and
use a value of about 10\%.

In order to find the angle of the panel, we'll want to reparameterize this in
spherical coordinates.
\begin{align*}
  \rho = \sqrt{x^2 + y^2 + z^2} \\
  \theta = \arctan\left(\frac{y}{x}\right) \\
  \phi = \arctan \frac{\sqrt{x^2 + y^2}}{z}
\end{align*}

These two angles will correspond to the direction the tracker is pointing.

\section{Using the sun data}

We can setup and calibrate the tracker using the highest position in the sky of
the sun, which will be at 1:00 PM on the day of the competition. This means the
tracker will have a permanently fixed axis at a downard angle of \(17.01\)
degrees, and will be pointed toward 197.9 degrees on a compass heading. We will
calibrate, if necessary, an offset for \(\theta\) that corresponds to \(\theta =
0\) in this position. Then, the only variable we have to worry about is
\(\phi\)! We can set the tracker to detect \(\theta\) as a quality control, and
use the magnetometer as another check.
\end{document}
